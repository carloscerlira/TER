\documentclass{article}
\usepackage[utf8]{inputenc}
\textheight = 25cm 
\textwidth = 15cm
\topmargin = -2.5cm 
\oddsidemargin = 1.5cm
\usepackage{float}
\usepackage{graphicx}
\graphicspath{{./images/}}

\usepackage{amsmath}
\usepackage{mathtools, xparse}
\usepackage[shortlabels]{enumitem}
\usepackage[most]{tcolorbox}
\usepackage{adjustbox}
\usepackage{bm} 

\DeclarePairedDelimiter{\norm}{\lVert}{\rVert}

\title{Termódinamica}
\author{Cerritos Lira Carlos}
\date{8 de Mayo del 2020}

\begin{document}
\maketitle
\section*{Entropía}
\subsubsection*{Motor ideal (Carnot)}
Tiene la mayor eficiencia, ningún motor real va poder 
superar su eficiencia. 
\subsubsection*{Segunda ley de la termodinámica}
No puede haber ningún motor cíclico cuyo único efecto sea bombear energía 
de un reservorio de calor y convertirla completamente en trabajo. \\ \\
Impone una limitación sobre la cantidad de trabajo que puede obtenerse de un 
motor, que opera entre dos fuentes(de alta temperatura y de baja temperatura). \\ \\
Clausius: el calor siempre se dirige de un cuerpo más caliente a un cuerpo más frío, 
el proceso inverso nunca se observa.
\subsubsection*{Entropía}
La entropía siempre aumenta en cualquier proceso espontáneo 
en un sistema aislado. \\ \\ 
Boltzman interpretación estádistica de la entropía
\subsubsection*{Teorema de Carnot}
Ningún motor térmico, funcionando cliclicamente entre dos focos térmicos fijos, tiene 
eficiencia mayor que un motor térmico reversible operando entre los mismos focos. 

\end{document}