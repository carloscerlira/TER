\documentclass{article}
\usepackage[utf8]{inputenc}
\textheight = 25cm 
\textwidth = 15cm
\topmargin = -3.0cm 
\oddsidemargin = 1.5cm
\usepackage{hyperref}
\hypersetup{
    colorlinks=true,
    linkcolor=blue,
    filecolor=blue,
    citecolor=black,      
    urlcolor=blue,
    }

\usepackage{float}
\usepackage{graphicx}

\usepackage{gensymb}

\usepackage{amsmath}
\usepackage{amssymb}
\usepackage{amsfonts}
\usepackage{mathtools, xparse}
\usepackage[shortlabels]{enumitem}

\usepackage[many]{tcolorbox}
\usepackage{lipsum}
\usepackage{amssymb}

\title{Tarea 3 Termodinámica}
\author{Cerritos Lira, Carlos}
\date{8 de junio del 2020}

\newcommand{\pr}[1]{\left(#1\right)}
\newcommand{\pt}[2]{\dfrac{\partial #1}{\partial #2}}

\begin{document}
\maketitle
\section*{1.-}
\subsection*{a)}
Encontrar las tres ecuaciones de estado del sistrema con ecuación fundamental
\[ u = (\tfrac{\theta}{R})s^2 - (\tfrac{R\theta}{v_0^2})v^2 \]
\begin{tcolorbox}
    Partimos de la relación:
    \begin{align*}
        U 
        &= (\tfrac{\theta}{R}) \tfrac{S^2}{N} - (\tfrac{R\theta}{v_0^2})\tfrac{V^2}{N} \\
        \tfrac{dU}{dS} 
        &= \tfrac{2\theta}{R}\tfrac{S}{N} \\
        &= T \\
        \tfrac{dU}{dV}
        &= -\tfrac{2R\theta}{v_0^2}\tfrac{V}{N} \\
        &= -p \\
        \tfrac{\partial U}{\partial N} 
        &= -(\tfrac{\theta}{R})\tfrac{S}{N^2}+(\tfrac{R\theta}{v_0^2})\tfrac{V^2}{N^2} \\
        &= \mu
    \end{align*}
\end{tcolorbox}

\subsection*{b)}
Mostrar que para este sistema $\mu = -u$
\begin{tcolorbox}
    \begin{align*}
        \tfrac{\partial U}{\partial N} &= -(\tfrac{\theta}{R})\tfrac{S}{N^2}+(\tfrac{R\theta}{v_0^2})\tfrac{V^2}{N^2} \\
        \mu &= -u    
    \end{align*}
\end{tcolorbox}

\subsection*{c)}
Expresar $\mu$ como función de $T$ y $p$
\begin{tcolorbox}
    \begin{align*}
        \mu &= \tfrac{Ts}{2} - \tfrac{pv}{2}
    \end{align*}
\end{tcolorbox}

\section*{2.-}
Se encuentra que un sistema obedece a las relaciones $U=PV$ y $P=BT^2$, donde $B$ es una constante.
Encontrar la función fundamental del sistema.
\begin{tcolorbox}
    Cálculamos las ecuaciones de estado del sistema:
    \begin{align*}
        u 
        &= BvT^2 \\
        \tfrac{1}{T}
        &= (\tfrac{Bv}{u})^{\frac{1}{2}} \\
        \tfrac{P}{T}
        &= \tfrac{u}{v}(\tfrac{Bv}{u})^{\frac{1}{2}} 
    \end{align*}
    Encontrmos $s(u,v)$ mediante la relación:
    \begin{align*}    
        ds
        &= \tfrac{1}{T}du + \tfrac{P}{T}dv \\
        &= (\tfrac{Bv}{u})^{\frac{1}{2}}du + \tfrac{u}{v}(\tfrac{Bv}{u})^{\frac{1}{2}}dv \\
        s(u,v)
        &= 2(\tfrac{Bv}{u})^{\frac{1}{2}}u + f(v) \\
        B(\tfrac{Bv}{u})^{-\frac{1}{2}} + f'(v) &= \tfrac{u}{v}(\tfrac{Bv}{u})^{\frac{1}{2}} \\
        f'(v) &= \tfrac{u}{v}(\tfrac{Bv}{u})^{\frac{1}{2}}-B(\tfrac{Bv}{u})^{-\frac{1}{2}} \\
        f'(v) &= 0 \\
        s(u,v) &= 2(\tfrac{Bv}{u})^{\frac{1}{2}}u+C 
    \end{align*}
    Multiplicando por $N$ encontramos la ecuación fundamental del sistema:
    \begin{align*}
        S(U,V,N) 
        &= Ns(u,v) \\
        &= 2(BVU)^{\frac{1}{2}}+CN 
    \end{align*}
\end{tcolorbox}

\end{document}