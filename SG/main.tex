\documentclass{article}
\usepackage[utf8]{inputenc}
\usepackage{amsmath}
\usepackage{amsfonts}
\usepackage{amssymb}
\usepackage{graphicx}
\usepackage{multirow}
\usepackage{tcolorbox}
\usepackage[left=2cm,right=2cm,top=2cm,bottom=2cm]{geometry}
\author{Ricky Frank}

\begin{document}

\section*{Primera ley}
Se tiene la relación:
\begin{align*}
    dU 
    &=dQ + dW \\
    &=dQ - PdV \\
    \therefore dQ 
    &=dU + PdV 
\end{align*}
Si suponemos que tenemos $U(T,V)$ se tiene:
\begin{align*}
    dU &=\frac{\partial U}{\partial T} dT + \frac{\partial U}{\partial V}dV 
\end{align*}
de donde obtenemos:
\begin{align*}
    dQ 
    &=dU + PdV \\
    &=\left(\frac{\partial U}{\partial T}\right)_V dT 
    + \left[\left(\frac{\partial U}{\partial V}\right)_T + P \right] dV
\end{align*}
se tiene entonces las relaciones:
\begin{align*}
    \left(\frac{\partial Q}{\partial T}\right)_V 
    &=\left(\frac{\partial U}{\partial T}\right)_V
\end{align*}
ahora bien, podemos obtener $\left(\frac{\partial Q}{\partial T}\right)_P$ mediante
la relación:
\begin{align*}
    Q(T,P) 
    &=Q(T,V(P,T)) \\
    \left(\frac{\partial Q}{\partial T}\right)_P
    &=\left(\frac{\partial U}{\partial T}\right)_V
    + \left[\left(\frac{\partial U}{\partial V}\right)_T + P \right] 
    \left( \frac{\partial V}{\partial T} \right)_P
\end{align*}
de donde obtenemos:
\begin{align*}
    C_P 
    &=C_V + \left[ \left( \frac{\partial U}{\partial V} \right)_T + P \right] V\beta \\
    \therefore \left( \frac{\partial U}{\partial V} \right)_T 
    &= \frac{C_P - C_V}{V \beta} - P
\end{align*}
\end{document}