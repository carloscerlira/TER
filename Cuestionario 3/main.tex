\documentclass{article}
\usepackage[utf8]{inputenc}
\textheight = 25cm 
\textwidth = 15cm
\topmargin = -3.0cm 
\oddsidemargin = 1.5cm
\usepackage{hyperref}
\hypersetup{
    colorlinks=true,
    linkcolor=blue,
    filecolor=blue,
    citecolor=black,      
    urlcolor=blue,
    }

\usepackage{float}
\usepackage{graphicx}

\usepackage{gensymb}

\usepackage{amsmath}
\usepackage{amssymb}
\usepackage{amsfonts}
\usepackage{mathtools, xparse}
\usepackage[shortlabels]{enumitem}

\usepackage[many]{tcolorbox}
\usepackage{lipsum}
\usepackage{amssymb}

\title{Cuestionario 3 Termodinámica}
\author{Cerritos Lira, Carlos}
\date{16 de junio del 2020}

\newcommand{\pr}[1]{\left(#1\right)}
\newcommand{\pt}[2]{\dfrac{\partial #1}{\partial #2}}

\begin{document}
\maketitle
\section*{1.-}
Dos sistemas tienen las siguientes ecuaciones de estado. 
\[ \frac{1}{T_1} = \frac{3}{2}R\frac{N_1}{U_1} \quad 
\frac{1}{T_2} = \frac{5}{2}R\frac{N_2}{U_2} \]
El número de moles del primer sistema es $N_1 = 2$ y del segundo $N_2 = 3$. Los dos 
sistemas se separan por una pared diatérmica. La energía total del sistema compuesto es $U_0$. 
¿Cuál es la energía de cada sistema y la temperatura en equilibrio?
\begin{tcolorbox}[breakable]
    Como estamos en equilibrio se cumple $T_1=T_2=T_0$, además:
    \begin{align*}
        U_1+U_2 &= U_0 \\
        \frac{3}{2}RN_1T_1 + \frac{5}{2}RN_2T_2 &= U_0 \\
        \frac{3}{2}R(2T_0+3T_0) &= U_0 \\
        T_0 &= \frac{2}{15}\frac{1}{R} U_0
    \end{align*}
    despejando obtenemos:
    \begin{align*}
        U_1 &= \frac{3}{15}\frac{1}{R}U_0 \\
        U_2 &= \frac{5}{15}\frac{1}{R}U_0
    \end{align*}
\end{tcolorbox}

\section*{2.-}
En la vecindad del estado $T_0,v_0$, el volumen de un sistema de un mol, se observa que varía 
de acuerdo con la relación:
\[ v = v_0 + a(T-T_0) + b(p-p_0) \]
Calcular la transferencia de calor $dQ$ del sistem si el volumen molar se cambia por un pequeño
incremento $dv = v-v_0$ a temperatura constante $T_0$
\begin{tcolorbox}[breakable]
    Notamos que tenemos $v(T,p)$. Como $T$ permanece constante:
    \begin{align*}
        dQ 
        &= Tds \\
        &= T\left(\frac{\partial s}{\partial T}dT + \frac{\partial s}{\partial v}dv \right), \quad \text{usando la representación de $s=s(T,v)$} \\
        &= T\frac{\partial s}{\partial v}dv, \quad \text{ya que $dT = 0$} \\
        &= T\frac{\partial p}{\partial T}dv, \quad \text{usando la relación de Maxwell para $\frac{\partial s}{\partial v}$}
    \end{align*}
    de la ecuación que satisface el sistema obtenemos:
    \begin{align*}
        p &= p_0 + \frac{1}{b}(v-v_0) - \frac{a}{b}(T-T_0) \\
        \frac{\partial p}{\partial T} &= -\frac{a}{b}
    \end{align*}
    sustituyendo en la primer ecuación encontrada obtenemos:
    \begin{align*}
        dQ
        &= -\frac{a}{b}Tdv
    \end{align*}
\end{tcolorbox}

\section*{3.-}
Obtener $\frac{\partial H}{\partial V}_{T,N}$ en términos de cantidades que se puedan medir en el 
laboratorio.
\begin{tcolorbox}[breakable]
    \begin{align*}
        H &= U + pV \\
        dH &= dU + Vdp + pdV \\
        dH &= TdS + Vdp \\
        \frac{\partial H}{\partial V} &= T\frac{\partial S}{\partial V} + V\frac{\partial p}{\partial V} \\
        \frac{\partial H}{\partial V} &= T\frac{\partial p}{\partial T} + V\frac{\partial p}{\partial V} 
    \end{align*}
\end{tcolorbox}

\end{document}