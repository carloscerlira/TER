\documentclass{article}
\usepackage[utf8]{inputenc}
\textheight = 25cm 
\textwidth = 15cm
\topmargin = -3.0cm 
\oddsidemargin = 1.5cm
\usepackage{hyperref}
\hypersetup{
    colorlinks=true,
    linkcolor=blue,
    filecolor=blue,
    citecolor=black,      
    urlcolor=blue,
    }

\usepackage{float}
\usepackage{graphicx}

\usepackage{gensymb}

\usepackage{amsmath}
\usepackage{amssymb}
\usepackage{amsfonts}
\usepackage{mathtools, xparse}
\usepackage[shortlabels]{enumitem}

\usepackage[many]{tcolorbox}
\usepackage{lipsum}
\usepackage{amssymb}

\title{Tarea Exámen Termodinámica}
\author{Cerritos Lira, Carlos}
\date{23 de junio del 2020}

\newcommand{\pr}[1]{\left(#1\right)}
\newcommand{\pt}[2]{\dfrac{\partial #1}{\partial #2}}

\begin{document}
\maketitle
\section*{1.-}
Imagina un filtro de aire especial colocado en la ventana de una casa. Los diminutos orificios en el filtro sólo
permiten la slaida de moléculas de aire cuya rapidez sea mayor que cierto valor, y sólo permiten la entrada a aquellas cuya 
rapidez sea menor que ese valor. Explica por qué tal filtro enfría la casa y por qué la segunda ley de la Termodinámica 
imposibilida la construcción de semejante filtro. 
\begin{tcolorbox}[breakable]

\end{tcolorbox}

\section*{2.-}
Decides tomar un reconfortante baño caliente, pero descubres tu que tu desconsiderado compañero de cuerto consumió cais toda el 
agua caliente. Entonces, llneas la tina con $280kg$ de agua a $30 \deg C$ e intentas calentarla más vertiendo $5kg$ más de agua 
que alcanzó la ebullición en una estufa.
\begin{enumerate}[a)]
    \item ¿Se trata de un proceso reversible o irreversible? Utiliza un razonamiento de física para explicar el hecho.
    \item Calcula la temperatura final dle agua para el baño.
    \item Calcula el cambio nto de la entropía del sistema (agua del baño + agua en ebullición), suponiendo
    que no hay intercambio de calor con el aire o con la tina misma.
\end{enumerate}
\begin{tcolorbox}[breakable]
    \subsubsection*{a)}
    
    \subsubsection*{b)}
    
    \subsubsection*{c)}
\end{tcolorbox}

\section*{3.-}
Te preparas un té con $0.250kg$ de agua a $85 \deg C$ y lo dejas enfriar a temperatura ambiente
$(20\deg C)$ antes de beberlo.
\begin{enumerate}[a)]
    \item Calcular el cambio de la entropía del agua mientras se enfria.
    \item En escencia, el proceso de enefriamento es isotérmico para el aire en tu cocina. Calcular el cambio 
    de entroía dle aire mientras el té se enfría, suponiendo que todo el calor que pierde el agua va al aire.
    ¿Cuál es el cambio de entroía del sistema constituido por té + aire?
\end{enumerate}
\begin{tcolorbox}[breakable]
    \subsubsection*{a)}

    \subsubsection*{b)}
\end{tcolorbox}

\section*{4.-}
Demostrar la relación:
\[ k_T = k_s + \frac{TV\beta^2}{C_p} \]
\begin{tcolorbox}[breakable]

\end{tcolorbox}

\section*{5.-}
Explciar de forma intuitiva ¿por qué $C_p \geq C_v$ y $k_t \geq k_s$?
\begin{tcolorbox}[breakable]

\end{tcolorbox}

\section*{6.-}
derivar una ecuación para la curva de coexistencia de fase entre el líquido y el gas, bajo la suposición de que el calor latente 
$L$ es independiente de la temperatura, que el vapor puede ser tratado como un gas ideal y que $V_{vap} >> V_{liq}$
\begin{tcolorbox}[breakable]

\end{tcolorbox}

\end{document}