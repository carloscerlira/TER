\documentclass{article}
\usepackage[utf8]{inputenc}
\textheight = 25cm 
\textwidth = 15cm
\topmargin = -3.0cm 
\oddsidemargin = 1.5cm
\usepackage{hyperref}
\hypersetup{
    colorlinks=true,
    linkcolor=blue,
    filecolor=blue,
    citecolor=black,      
    urlcolor=blue,
    }

\usepackage{float}
\usepackage{graphicx}

\usepackage{gensymb}

\usepackage{amsmath}
\usepackage{amssymb}
\usepackage{amsfonts}
\usepackage{mathtools, xparse}
\usepackage[shortlabels]{enumitem}

\usepackage[many]{tcolorbox}
\usepackage{lipsum}
\usepackage{amssymb}

\title{Tarea Exámen Termodinámica}
\author{Cerritos Lira, Carlos}
\date{23 de junio del 2020}

\newcommand{\pr}[1]{\left(#1\right)}
\newcommand{\pt}[2]{\dfrac{\partial #1}{\partial #2}}

\begin{document}
\maketitle
\section*{1.-}
Imagina un filtro de aire especial colocado en la ventana de una casa. Los diminutos orificios en el filtro sólo
permiten la salida de moléculas de aire cuya rapidez sea mayor que cierto valor, y sólo permiten la entrada a aquellas cuya 
rapidez sea menor que ese valor. Explica por qué tal filtro enfría la casa y por qué la segunda ley de la Termodinámica 
imposibilida la construcción de semejante filtro. 
\begin{tcolorbox}[breakable]
    La casa se enfría ya que después de un tiempo la energía cinética de las moléculas dentro solo puede ser menor 
    que su energía cinética promedio al inicio, por ende la temperatura final es menor. \\ \\
    Dicho filtro no se puede contruir, ya que para clasificar las moléculas necesitaría suministrar 
    energía, es decir este no es un proceso espontáneo, de lo contrario violaría la segunda ley de la termodinámica.
\end{tcolorbox}

\section*{2.-}
Decides tomar un reconfortante baño caliente, pero descubres que tu desconsiderado compañero de cuarto consumió casi toda el 
agua caliente. Entonces, llenas la tina con $280kg$ de agua a $30 \deg C$ e intentas calentarla más vertiendo $5kg$ más de agua 
que alcanzó la ebullición en una estufa.
\begin{enumerate}[a)]
    \item ¿Se trata de un proceso reversible o irreversible? Utiliza un razonamiento de física para explicar el hecho.
    \item Calcula la temperatura final del agua para el baño.
    \item Calcula el cambio neto de la entropía del sistema (agua del baño + agua en ebullición), suponiendo
    que no hay intercambio de calor con el aire o con la tina misma.
\end{enumerate}
\begin{tcolorbox}[breakable]
    \subsubsection*{a)}
    Irreversible, ya que sería necesario realizar trabajo sobre el sistema
    para poder regresar al estado inicial.

    \subsubsection*{b)}
    Tenemos la relación:
    \begin{align*}
        Q &= Q_1+Q_2 \\ 
        0 &= m_1c\Delta T_1+m_2c\Delta T_2 \\
        0 &= 270c(T-30)+5c(T-100) \\
        275T &= 270(30)+5(100) \\
        T &= \frac{8600}{275} \\
        T &= 31.27 \deg C = 304.42 K
    \end{align*}
    \subsubsection*{c)}
    Partimos de la relación:
    \begin{align*}
        \Delta S 
        &= \Delta S_1 + \Delta S_2 \\
        &= m_1cln(\tfrac{304.42K}{303.15K}) + m_2cln(\tfrac{304.42K}{373.15K}) \\
        &= 250kg \times 4190 \tfrac{J}{kg \cdot K} \times ln(\tfrac{304.42K}{303.15K}) + 5kg \times \times 4190 \tfrac{J}{kg \cdot K} \times ln(\tfrac{304.42K}{373.15K}) \\
        &= 464.66 \tfrac{J}{K}
    \end{align*}
\end{tcolorbox}

\section*{3.-}
Te preparas un té con $0.250kg$ de agua a $85 \deg C$ y lo dejas enfriar a temperatura ambiente
$(20\deg C)$ antes de beberlo.
\begin{enumerate}[a)]
    \item Calcular el cambio de la entropía del agua mientras se enfria.
    \item En escencia, el proceso de enefriamento es isotérmico para el aire en tu cocina. Calcular el cambio 
    de entropía del aire mientras el té se enfría, suponiendo que todo el calor que pierde el agua va al aire.
    ¿Cuál es el cambio de entrpía del sistema constituido por té + aire?
\end{enumerate}
\begin{tcolorbox}[breakable]
    \subsubsection*{a)}
    Nuevamente partimos de la relación:
    \begin{align*}
        \Delta S_{te}
        &= mcln(\tfrac{T_2}{T_1}) \\
        &= 0.25kg \times 4190 \tfrac{J}{kg \cdot K} \times ln(\tfrac{293K}{358K}) \\
        &= -210\tfrac{J}{K}
    \end{align*}
    \subsubsection*{b)}
    De acuerdo a la conservación de la energía el calor perdido por el agua 
    es absorbodo por el aire, esto es:
    \begin{align*}
        Q_{aire} 
        &= -Q_{te} \\
        &= -mc\Delta T \\
        &= -mc(T_2-T_1) \\
        &= -0.25kg \times 4100 \tfrac{J}{kg \cdot K} \times (293K-358K) \\
        &= 6.8 \times 10^4 J
    \end{align*}
    Ahora bien, como el proceso de enfriamento es isotérmico, podemos encontrar 
    el cambio de entropía del aire por:
    \begin{align*}
        \Delta S_{aire} 
        &= \tfrac{Q_{aire}}{T_{aire}} \\
        &= \tfrac{6.8 \times 10^4J}{293 K} \\
        &= 232.1 \tfrac{J}{K} 
    \end{align*}
    El cambio de entrpía total está dado por:
    \begin{align*}
        \Delta S 
        &= \Delta S_{aire} + \Delta S_{te} \\
        &= -210 \tfrac{J}{K} + 232.1 \tfrac{J}{K} \\
        &= 22.1 \tfrac{J}{K}
    \end{align*}
\end{tcolorbox}

\section*{4.-}
Demostrar la relación:
\[ k_T = k_s + \frac{TV\beta^2}{C_p} \]
\begin{tcolorbox}[breakable]
    Por definición:
    \begin{align*}
        C_p &= T(\tfrac{\partial S}{\partial T})_p \\
        C_V &= T(\tfrac{\partial S}{\partial T})_V
    \end{align*}
    rescribimos $C_p$:
    \begin{align*}
        C_p 
        &= T(\tfrac{\partial S}{\partial T})_p \\
        &= T(\tfrac{\partial S}{\partial T})_V + T(\tfrac{\partial S}{\partial V})_T (\tfrac{\partial V}{\partial T})_p \\
        &= C_V + TV\beta(\tfrac{\partial S}{\partial V})_T && \text{usando la definición $\beta = \tfrac{1}{V}(\tfrac{\partial V}{\partial T})_p$} \\
        &= C_V + \tfrac{TV\beta^2}{k_T} && \text{usando la relación $(\tfrac{\partial S}{\partial V})_T = \tfrac{\beta}{k_T}$, obtenida usando Maxwell}
    \end{align*}
    ahora bien, observamos que si se cumple:
    \begin{align*}
        \frac{C_p}{C_V} &= \frac{k_T}{k_s}
    \end{align*}
    ya habríamos acabdo la demostración, veremos que esto se satisface. \\
    Usando la relación cíclica:
    \begin{align*}
        (\tfrac{\partial S}{\partial T})_p (\tfrac{\partial T}{\partial p})_S (\tfrac{\partial p}{\partial S})_T &= -1 \\
        (\tfrac{\partial S}{\partial T})_p &= -(\tfrac{\partial p}{\partial T})_S (\tfrac{\partial S}{\partial p})_T
    \end{align*}
    de la misma manera:
    \begin{align*}
        (\tfrac{\partial S}{\partial T})_V(\tfrac{\partial T}{\partial V})_S (\tfrac{\partial V}{\partial S})_T &= -1 \\
        (\tfrac{\partial S}{\partial T})_V &= -(\tfrac{\partial V}{\partial T})_S (\tfrac{\partial S}{\partial V})_T
    \end{align*}
    sustituyendo encontrmaos:
    \begin{align*}
        \tfrac{C_p}{C_V}
        &= \frac{(\tfrac{\partial S}{\partial T})_p}{(\tfrac{\partial S}{\partial T})_V} \\
        &= \frac{(\tfrac{\partial p}{\partial T})_S (\tfrac{\partial S}{\partial p})_T}{(\tfrac{\partial V}{\partial T})_S (\tfrac{\partial S}{\partial V})_T} \\
        &= \frac{(\tfrac{\partial V}{\partial p})_T}{(\tfrac{\partial V}{\partial p})_S} \\
        &= \frac{k_T}{k_s}
    \end{align*}
    Sutituyendo la relación encontrada en la ecuación para $C_p$:
    \begin{align*}
        C_p &= C_V + \tfrac{TV\beta^2}{k_T} \\
        1 &= \tfrac{C_V}{C_p} + \tfrac{TV\beta^2}{C_pK_T} \\
        1 &= \tfrac{k_s}{k_T} + \tfrac{TV\beta^2}{C_pK_T} \\
        k_T &= k_s + \tfrac{TV\beta^2}{C_p}
    \end{align*}
    que es lo que se quería demostrar.
\end{tcolorbox}

\section*{5.-}
Explciar de forma intuitiva ¿por qué $C_p \geq C_v$ y $k_t \geq k_s$?
\begin{tcolorbox}[breakable]
    El calor específico es la cantidad de calor que se requiere para elevar la emperatura una unidad. 
    Si añadimos este calor manteniendo el volumen constante $C_p$, todo la energía se usa para elevar la energía cinética de las partículas, 
    sin embargo si añadimos este calor a presión constante $C_V$, a la hora de expanderse el gas debe de realizar un trabajo sobre las paredes del 
    contenedor, por esta razón $C_p \geq C_v$, y puede demostrar que cumplen la relación:
    \[ C_p = C_v + R \]

\end{tcolorbox}

\section*{6.-}
Derivar una ecuación para la curva de coexistencia de fase entre el líquido y el gas, bajo la suposición de que el calor latente 
$L$ es independiente de la temperatura, que el vapor puede ser tratado como un gas ideal y que $V_{vap} >> V_{liq}$
\begin{tcolorbox}[breakable]
Usamos la ecuación de Clausius–Clapeyron que aplica a todas las curvas de coexistencia:
\begin{align*}
    \frac{dP}{dT} 
    &= \frac{L(T)}{T\Delta V(T)} \\
    &= \frac{L}{TV_{vap}} && \text{Dado que $V_{vap} >> V_{liq}$ y $L$ no depende de $T$} \\
    &= \frac{LP}{RT^2} && \text{Usando la ley del gas ideal $V = \frac{RT}{P}$} 
\end{align*}
de donde obtenemos la ecuación diferencial:
\begin{align*}
    \frac{\tfrac{dP}{dT}}{P} 
    &= \frac{L}{RT^2} \\ 
    \frac{d}{dT}ln(P(T)) &= \frac{L}{RT^2} \\
    ln(P(T)) &= -\frac{L}{RT}+C \\
    P(T) &= Kexp(-\tfrac{L}{RT})  
\end{align*}
la cual es la curva de coexistencia de la fase líqudo-gas.
\end{tcolorbox}

\end{document}