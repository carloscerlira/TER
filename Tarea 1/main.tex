\documentclass{article}
\usepackage[utf8]{inputenc}
\textheight = 25cm 
\textwidth = 15cm
\topmargin = -3.0cm 
\oddsidemargin = 1.5cm
\usepackage{hyperref}
\hypersetup{
    colorlinks=true,
    linkcolor=blue,
    filecolor=blue,
    citecolor=black,      
    urlcolor=blue,
    }

\usepackage{float}
\usepackage{graphicx}

\usepackage{gensymb}

\usepackage{amsmath}
\usepackage{amssymb}
\usepackage{amsfonts}
\usepackage{mathtools, xparse}
\usepackage[shortlabels]{enumitem}

\usepackage[many]{tcolorbox}
\usepackage{lipsum}

\title{Tarea 2 Matemáticas Avanzadas de la Física}
\author{Cerritos Lira Carlos}
\date{2 de Marzo del 2020}

\begin{document}
\maketitle
\section*{1.-}
Las isotermas de un sistema pueden cortarse? Explique detalladamente.

\section*{2.-}
Sean $A,B,C$ tres gases con variables $(p,V),(p',V'),(p'',V'')$, respectivamente. 
Cuando $A$ está en equilibrio térmico con $C$ se satisface la ecuación:
\begin{align*}
    pV - nbp - p''V'' = 0
\end{align*}
cuando $B$ está en equilibrio térmico con $C$ se cumple:
\begin{align*}
    p'V' - p''V'' + \frac{nB'p''V''}{V'} &= 0
\end{align*}
\begin{enumerate}[a)]
    \item Encuentre las funciones $q(p,V),r(p',V')$ y $s(p'',V'')$ que son iguales 
    entre si en el equilibrio térmico e iguales a la temperatura común $T$.

    \item Encuentre la ecuación que corresponde a $A$ en equilibrio térmico con $B$
\end{enumerate}

\section*{3.-}
Un gas ideal se caracteriza mediante dos suposiciones: los átomos o moléculas de un gas 
ideal no interactúan entre sí, y los átomos o moléculas se pueden tratar como masas 
puntuales. Esto tiene un rango de validez limitada. \\ \\ 
Se muestra la energía potencial de interacción de dos moléculas de gas en función 
de la distancia entre ellas. El potencial intermolecular se peude dividir en regiones
en las que la energía potencial es esencialmente nula $(r>r_{trans})$, negativa
(interacción atractiva) $(r_{trans} > r > r_{V=0})$, y positiva (interacción repulsiva).
La distancia $r_{trans}$ no está defininida unívocamente y depende de la energía de la 
molécula.

\begin{enumerate}[a)]
    \item Calcule la presión ejercida por $N_2$ a $300K$ para vólumenes molares de 
    $250$ y $0.100L$ usando las ecuaciones de estado del gas ideal y de van der Waals.
    Los valores de los parámetros $a$ y $b$ para $N_2$ son $1.360bar dm^3 mol^{-2}$ y
    $0.0387 dm^3 mol^{-1}$ respectivamente. 
    
    \item Compare y explique los reusltados de los cálculos a las dos presiones y 
    explique qué predomina, las intersacciones atractivas o las repulsivas.

    \item A temperatura suficientemente elevada, la ecuación de van der Waals tiene 
    la forma $P = \frac{nR}{V-b}$. Nótese que la parte atractiva del potencial no tiene 
    influencia en esta expresión. Justifique este comportamiento uisando el diagrama 
    anterior.
\end{enumerate}

\section*{4.-}
Sabemos que la ecuación de van der Waals permite hacer una descripción cualitativa 
de la transición líquido- vapor, por lo cual, como se mencionó en clase, el 
valor del parámetro $a$ tiene una relación con el calor de vaporzación del líquido y 
el párametro $b$  con el tamaño de las moléculas. Del artículo de van der Waals haga 
una descripción, en no menos de una cuartilla, de lo que son las fuerzas de 
van der Waals, como se analiza en el punto crítico, el factor de compresibildiad y al 
ensión superficial.

\section*{5.-}
Sea un gas en un recipiente cilíndrico de un metro de altura, cuya tapa superior es 
un pistón inicialmente fijo. El gas está en equilibrio a temperatura y presión ambiente
y bajo estas condiciones sabemos que la velocidad de propragación del sonido es 
$c = 340 \frac{m}{seg}$. Supongamso que liberamos al pistón para que pueda moverse 
libremente. Ahora le damos un martillazo al pistón, generando así una situación de
desequilibrio en el gas. También inmediatamente aislamos térmicamente al sistema. 
\begin{enumerate}[a)]
    \item Culcule el tiempo que tarda en saber la parte inferior del gas que ha sido 
    golpeado 

    \item Discuta cualitativamente qué ocurrió con la energía proporcionada por el 
    martillazo. 

    \item ¿Se restaura el equilibrio o queda oscilando?
\end{enumerate}

\section*{6.-}
Demuestre que para un gas ideal $pv = RT$, $\beta = \frac{1}{T}$, $k = \frac{1}{p}$. \\
Para un gas real a presiones moderadas, $P(v-b) = RT$, donde $R$ y $b$ son constantes, 
es una ecuación de estado aproximada que tiene en cuenta el tamaño finito de las moléculas.
Demostrar que:
\begin{enumerate}[a)]
    \item $\beta = \frac{\frac{1}{T}}{1+\frac{bP}{RQ}}$
    \item $k = \frac{\frac{1}{p}}{1+\frac{bP}{RT}} $
\end{enumerate}

\section*{7.-}
Un hilo metálico de $0.0085cm^3$ de sección transversal está sometido a una tensión de 
$20N$, a la temperatura de $10\degree C$, ente dos soportes rígidos separados $1.2m$. 
\begin{enumerate}[a)]
    \item ¿Cuál es la tensión final, si la temperatura se reduce $8\degree C$?
    $(\alpha = 1.5 \times 10^{-5}K^{-1}), Y = 2.0 \times 10^9 \frac{N}{m^2}$

    \item La frecuencia fundamental de vibración de un alambre de longitud $L$, 
    masa $m$ y tensión $\tau$ es:
    \[ f_1 = \frac{1}{2L} \sqrt{\frac{\tau L}{m}} \]
    ¿Con qué frecuencia vibra el hilo a $20\degree C$. ¿Y a $8\degree C$
\end{enumerate}

\section*{8.-}
Muestre que si las diferenciales $dV, dp$ dadas por:
\begin{align*}
    dV &= v\beta dT - Vk_T dp \\
    dp &= \frac{\beta}{k_T} dT - \frac{1}{V_{k_T}}dV
\end{align*}
Son exactas, entonces los coeficientes satisfacen las siguientes relaciones:
\begin{align*}
    \left( \frac{\partial V\beta}{\partial p} \right)
    &=\left( \frac{\partial V_{k_T}}{\partial T} \right) \\
    - \left( \frac{\partial \beta}{\partial V} \right)
    &=\beta \left( \frac{\partial lnK_T}{\partial V} \right) 
    + \frac{1}{V}\left( \frac{\partial lnk_t}{\partial T} \right)
\end{align*}
\section*{9.-} 
La resistencia eléctrica en el hilo de un termómetro de platino varía linealmente con
la temperatura. Determinar:
\begin{enumerate}
    \item La expresión de la temperatura centígrada en el punto de fusión del hielo 
    $R_0$ y en el punto de ebullición del agua $R_{100}$. 
    
    \item Si los valores de una resistencia para un termómetro de hilo de platino son de 
    $R_0 = 10000 \Omega$ y $R_{100} = 13861 \Omega$, calcular la temperatura correspondiente 
    a una resistencia de $26270 \Omega$. 
\end{enumerate}



\end{document}